\documentclass[12pt]{report}

\usepackage[russian]{babel}
\usepackage[utf8]{inputenc}
\usepackage{amsmath}
\usepackage{amssymb}
\usepackage{tikz}
\usepackage{caption}
\usepackage[a4paper,margin=1.0in,footskip=0.25in]{geometry}
\usepackage{listings}  
\usepackage{color}
\usepackage{tabto}

\definecolor{dkgreen}{rgb}{0,0.6,0}
\definecolor{gray}{rgb}{0.5,0.5,0.5}
\definecolor{mauve}{rgb}{0.58,0,0.82}

\lstset{frame=tb,
  language=Python,
  aboveskip=3mm,
  belowskip=3mm,
  showstringspaces=false,
  columns=flexible,
  basicstyle={\small\ttfamily},
  numbers=none,
  numberstyle=\tiny\color{gray},
  keywordstyle=\color{blue},
  commentstyle=\color{dkgreen},
  stringstyle=\color{mauve},
  breaklines=true,
  breakatwhitespace=true,
  tabsize=3,
  frame=none
}

%Title
\title{\vspace{-3cm}Лабораторная №2}
\author{Жидович Максим, группа №1}
\date{6 октября 2021}

\renewcommand\thesection{\arabic{section}}

\begin{document}

\maketitle

\section{Постановка задачи}

Разработать программу численного решения СЛАУ на основе $LDL^{T}$-разложения
\[
\begin{bmatrix}
a_{11} & a_{12}  & \cdots   & a_{1m}   \\
a_{21} & a_{22}  & \cdots   & a_{2m}  \\
\vdots & \vdots  & \ddots   & \vdots  \\
a_{n1} & a_{n2}  & \cdots\  & a_{nm}  \\
\end{bmatrix}
\begin{bmatrix}
x_{1} \\
x_{2} \\
\vdots \\
x_{n} \\
\end{bmatrix}
=
\begin{bmatrix}
b_{1} \\
b_{2} \\
\vdots \\
b_{n} \\
\end{bmatrix}
\]

\section{Входные данные}

В работе использовались:

Матрица $A_{1}$ размерности 5х5 сгенерированная случайным образом при $k = 0$:
\[
\begin{bmatrix}
      11 &        -2 &        -4 &        -3 &        -1 & \\
        -2 &         4 &         0 &        -1 &        -1 & \\
        -4 &         0 &         4 &         0 &         0 & \\ 
        -3 &        -1 &         0 &         8 &        -4 & \\ 
        -1 &        -1 &         0 &        -4 &         6 & \\
\end{bmatrix}
\]

Матрица $A_{2}$ размерности 5х5 сгенерированная случайным образом при $k = 1$:
\[
\begin{bmatrix}
      10.1 &        -2 &        -4 &        -3 &        -1 & \\
        -2 &         4 &         0 &        -1 &        -1 & \\ 
        -4 &         0 &         4 &         0 &         0 & \\ 
        -3 &        -1 &         0 &         8 &        -4 & \\ 
        -1 &        -1 &         0 &        -4 &         6 & \\
\end{bmatrix}
\]

Вектор $x$, полученный при $m = 4$ и $n = 5$:
\[
\begin{bmatrix}
   4 & 5 & 6 & 7 & 8   
\end{bmatrix}
\]

Вектора $b_{1}, b_{2}$, полученные умножением $A_{1}, A_{2}$ на $x$:
\[
\begin{bmatrix}
   -19 & -3 & 8 & 7 & 11
\end{bmatrix}
\]
\[
\begin{bmatrix}
   -22.6 & -3 & 8 & 7 & 11  
\end{bmatrix}
\]

\section{Краткая теория}

Mатрица $L$ - нижняя треугольная (за исключением единиц на главной диагонали), хранится на месте нижнего треугольника матрицы $A$, диагональная матрица $D$ хранится на месте главной диагонали матрицы $A$.

Используется алгоритм (5) файла «LDLtRtR разложения», требующий хранения только нижнего треугольника матрицы.

\section{Листинг программы}

\lstset{language=Python}
\lstset{extendedchars=\true}

Код программы, реализующий $LDL^{T}$-разложение: 

\begin{lstlisting}
def LDLt(m):
    """Find LDLt decomposition"""
    a = [i.copy() for i in m]
    n = len(a)
    t = [0] * (n**2)

    # transformate matrix A
    for k in range(n - 1):
        for i in range(k + 1, n):
            t[i] = a[i][k]
            a[i][k] /= a[k][k]
            for j in range(k + 1, i + 1):
                a[i][j] -= a[i][k] * t[j]

    # find matrix L
    l = [[0] * n for i in range(n)]
    for i in range(n):
        l[i][i] = 1
    for i in range(1, n):
        for j in range(i):
            l[i][j] = a[i][j]

    # find matrix D
    d = [[0] * n for i in range(n)]
    for i in range(n):
        d[i][i] = a[i][i]

    return (l, d, transpose(l))
\end{lstlisting}

Код программы, реализующий нахождение решения с помощью $LDL^{T}$-разложения: 

\begin{lstlisting}
def solve(a, b):
    """Solve A*x = b using LDLt decomposition"""
    l, d, l_t = LDLt(a)
    n = len(a)
    x = [0] * n

    # solve L*Y=b
    y = [0] * n
    for i in range(n):
        y[i] = b[i]
        for j in range(i):
            y[i] -= l[i][j] * y[j]

    # solve D*Lt*x= b
    for i in range(n - 1, -1, -1):
        x[i] = y[i] / d[i][i]
        for j in range(n - 1, i, -1):
            x[i] -= l[j][i] * x[j]

    return x
\end{lstlisting}

\section{Выходные данные}

\hspace{-0.5cm}$L_{1} = 
\begin{bmatrix}
       1 &         0 &         0 &         0 &         0 & \\
-0.18181818181818182 &         1 &         0 &         0 &         0 & \\
-0.36363636363636365 &      -0.2 &         1 &         0 &         0 & \\
-0.2727272727272727 &    -0.425 &-0.5833333333333334 &         1 &         0 & \\
-0.09090909090909091 &    -0.325 &-0.25000000000000006 &-0.897810218978102 &         1 & \\
\end{bmatrix}
$

\vspace{0.5cm}\hspace{-0.2cm}$D_{1} = 
\begin{bmatrix}
     11 &         0 &         0 &         0 &         0 & \\
         0 &3.6363636363636362 &         0 &         0 &         0 & \\
         0 &         0 &       2.4 &         0 &         0 & \\
         0 &         0 &         0 &5.708333333333334 &         0 & \\
         0 &         0 &         0 &         0 &0.7737226277372278 & \\
\end{bmatrix}
$

\vspace{0.5cm}\hspace{-2.3cm}$L_{2} = 
\begin{bmatrix}
       1 &         0 &         0 &         0 &         0 & \\
-0.19801980198019803 &         1 &         0 &         0 &         0 & \\
-0.39603960396039606 &-0.21978021978021978 &         1 &         0 &         0 & \\
-0.297029702970297 &-0.44230769230769235 &-0.6862745098039215 &         1 &         0 & \\
-0.09900990099009901 &-0.3324175824175824 &-0.2941176470588236 &-0.9871677360219983 &         1 & \\
\end{bmatrix}
$

\vspace{0.5cm}\hspace{-1.6cm}$D_{2} = 
\begin{bmatrix}
     10.1 &         0 &         0 &         0 &         0 & \\
         0 &3.603960396039604 &         0 &         0 &         0 & \\
         0 &         0 &2.241758241758242 &         0 &         0 & \\
         0 &         0 &         0 &5.348039215686275 &         0 & \\
         0 &         0 &         0 &         0 &0.09715857011915485 & \\
\end{bmatrix}
$

\vspace{0.5cm}\hspace{-2.6cm}$x_{1} = 
\begin{bmatrix}
     3.9999999999999885 & 4.999999999999987 & 5.9999999999999885 & 6.999999999999986 & 7.999999999999986
\end{bmatrix}
$

\vspace{0.5cm}\hspace{-2.6cm}$x_{2} = 
\begin{bmatrix}
     4.000000000000133 & 5.000000000000135 & 6.000000000000133 & 7.000000000000136 & 8.000000000000137
\end{bmatrix}
$

\vspace{0.5cm}Относительная огрешность полученная при $k = 0: 1.7763568394002505e-15$

Относительная огрешность полученная при $k = 1: 1.709743457922741e-14$

\section{Выводы}

\tabПогрешности близки к нулю (порядок $10^{-15}$), это говорит о том, что ответ достаточно точен.

\end{document}
